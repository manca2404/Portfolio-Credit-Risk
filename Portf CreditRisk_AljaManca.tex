\documentclass[12pt,a4paper]{article}

\usepackage[utf8]{inputenc}
\usepackage[T1]{fontenc}
\usepackage{amsmath}
\usepackage{amssymb}
\usepackage{graphicx}
\usepackage{booktabs}
\usepackage{array}
\usepackage{geometry}
\usepackage{setspace}
\usepackage{hyperref}

\geometry{margin=1in}
\setstretch{1.2}

\begin{document}

\begin{titlepage}
    \centering
    {\large University of Ljubljana\\[0.5cm]
    School of Economics and Business \par}
    \vspace{5cm}
    {\LARGE \textbf{Portfolio Credit Risk Models:\\A Comprehensive Review}\par}
    \vspace{10cm}
    {\large Risk Management 2025/26\\[0.2cm]
    Author: Alja Dostal, Manca Kav\v{c}i\v{c}\\[0.2cm]
    Date: \today\par}
\end{titlepage}

% ------------------------
% TABLE OF CONTENTS
% ------------------------
\tableofcontents
\newpage

% ------------------------
% MAIN TEXT
% ------------------------

\begin{abstract}
This paper provides a comprehensive overview of portfolio credit risk modelling 
frameworks, including structural, reduced-form, actuarial, copula-based, and 
macro-econometric approaches. It presents a taxonomy of models, discusses 
theoretical foundations, evaluates practical applications, and highlights key 
strengths and limitations. A comparative table summarises the major portfolio 
credit risk models widely used in practice.
\end{abstract}

Credit risk---the potential loss arising from a counterparty's failure to meet 
contractual obligations---remains one of the central risks in financial 
institutions. While single-obligor analysis focuses on the creditworthiness of 
individual borrowers, portfolio credit risk concerns the joint behaviour of many 
borrowers, correlations between defaults, and the distribution of potential 
losses.
borrowers, correlations between defaults, and the distribution of potential 
losses.

Portfolio credit risk modelling provides estimates of expected loss (EL), 
unexpected loss (UL), and tail risk measures such as Value-at-Risk (VaR) and 
Expected Shortfall (ES). These measures are essential for economic capital, 
credit portfolio optimisation, stress testing, and regulatory compliance under 
Basel II and Basel III.

This paper provides a structured overview of the most widely used modelling 
frameworks and assesses their strengths and weaknesses.

\section{Conceptual Foundations of Portfolio Credit Risk}

Portfolio credit risk models aim to quantify the loss distribution:
\[
L = \sum_{i=1}^{n} EAD_i \cdot LGD_i \cdot D_i,
\]
where $D_i$ is a default indicator. The main challenge is modelling the joint 
distribution of defaults and exposures.

Key modelling dimensions include:
\begin{itemize}
    \item top-down vs.\ bottom-up approaches,
    \item default-mode vs.\ mark-to-market models,
    \item conditional vs.\ unconditional models,
    \item structural vs.\ reduced-form models.
\end{itemize}

Each modelling philosophy captures different aspects of credit portfolio behaviour.

\section{Typology of Portfolio Credit Risk Models}

\subsection{Top-Down vs.\ Bottom-Up Models}

\subsubsection{Bottom-Up Models}
Bottom-up models describe credit risk at the obligor level and aggregate results 
to the portfolio. Examples include the Merton/KMV model, CreditMetrics, 
Gaussian copula models, and CreditRisk+.

Advantages:
\begin{itemize}
    \item granular risk insights,
    \item suitable for optimisation,
    \item flexible dependence modelling.
\end{itemize}

Disadvantages:
\begin{itemize}
    \item computationally intensive,
    \item require detailed obligor-level data.
\end{itemize}

\subsubsection{Top-Down Models}
Top-down approaches treat the portfolio as a whole, without modelling individual 
obligors. Examples include Poisson cluster models and loss distribution 
approaches.

Advantages:
\begin{itemize}
    \item computational efficiency,
    \item useful for stress testing.
\end{itemize}

Disadvantages:
\begin{itemize}
    \item limited interpretability,
    \item no borrower-level insights.
\end{itemize}

\subsection{Default-Mode vs.\ Mark-to-Market Models}

\subsubsection{Default-Mode Models}
Default-mode models compute losses only when default occurs. Examples include 
CreditRisk+, copula-based models, and factor models such as Vasicek.

Strengths:
\begin{itemize}
    \item conceptual simplicity,
    \item efficient computation.
\end{itemize}

Weaknesses:
\begin{itemize}
    \item ignore credit migrations,
    \item limited for pricing and mark-to-market analysis.
\end{itemize}

\subsubsection{Mark-to-Market Models}
Mark-to-market models consider changes in credit quality prior to default. 
CreditMetrics and CreditPortfolioView are key examples.

Strengths:
\begin{itemize}
    \item capture migrations and spread risk,
    \item more complete valuation framework.
\end{itemize}

Weaknesses:
\begin{itemize}
    \item require more data,
    \item more complex to calibrate.
\end{itemize}

\subsection{Conditional vs.\ Unconditional Models}

Unconditional models assume fixed PDs, while conditional models link default 
probabilities to macroeconomic or latent factors.

Examples of conditional models: CreditPortfolioView, Vasicek, and copula models 
with systematic factors.

\subsection{Structural Models}

Structural models originate from Merton (1974). Firms default when asset values 
fall below a barrier. Asset values typically follow geometric Brownian motion.

Advantages:
\begin{itemize}
    \item strong economic foundation,
    \item market-implied PDs.
\end{itemize}

Disadvantages:
\begin{itemize}
    \item unrealistic default timing,
    \item poor applicability to private firms.
\end{itemize}

\subsection{Reduced-Form (Intensity) Models}

Reduced-form models treat default as a jump process with stochastic intensity.

Advantages:
\begin{itemize}
    \item flexible and tractable,
    \item compatible with interest rate models.
\end{itemize}

Disadvantages:
\begin{itemize}
    \item no structural interpretation,
    \item calibration requires extensive market data.
\end{itemize}

\section{Major Portfolio Credit Risk Models}

\subsection{CreditMetrics}

CreditMetrics (J.P.\ Morgan, 1997) is a market-to-market credit risk model based 
on rating transitions, spread changes, and correlated asset value dynamics. Loss 
arises from both defaults and rating migrations.

Strengths:
\begin{itemize}
    \item captures migrations,
    \item consistent with market risk frameworks.
\end{itemize}

Weaknesses:
\begin{itemize}
    \item correlation estimation is difficult,
    \item reliance on ratings may lag market reality.
\end{itemize}

\subsection{KMV / Merton Structural Model}

The distance-to-default (DD) is defined as:
\[
DD = \frac{\ln(V/A) + (\mu - \tfrac{1}{2}\sigma^2)T}{\sigma\sqrt{T}}.
\]

KMV maps DD to empirical default probabilities.

Strengths:
\begin{itemize}
    \item grounded in firm-value economics,
    \item provides forward-looking PDs.
\end{itemize}

Weaknesses:
\begin{itemize}
    \item requires market data,
    \item default barriers difficult to identify.
\end{itemize}

\subsection{CreditRisk+}

CreditRisk+ is an actuarial model treating defaults as Poisson events with 
sector-driven intensities.

Strengths:
\begin{itemize}
    \item highly efficient,
    \item minimal data requirements.
\end{itemize}

Weaknesses:
\begin{itemize}
    \item ignores rating migrations,
    \item Poisson assumption may oversimplify dependence.
\end{itemize}

\subsection{Gaussian Copula Models}

These models use copulas to model dependence between defaults:
\[
D_i = \mathbf{1}(Z_i < \Phi^{-1}(PD_i)).
\]

Strengths:
\begin{itemize}
    \item flexible dependence structure,
    \item widely used in CDO pricing.
\end{itemize}

Weaknesses:
\begin{itemize}
    \item Gaussian copula underestimates tail dependence,
    \item contributed to mispricing before 2008.
\end{itemize}

\subsection{CreditPortfolioView}

CreditPortfolioView links default probabilities to macroeconomic variables such 
as GDP, unemployment, interest rates, and inflation.

Strengths:
\begin{itemize}
    \item suitable for stress testing,
    \item offers economic interpretation.
\end{itemize}

Weaknesses:
\begin{itemize}
    \item requires long macroeconomic series,
    \item calibration is complex.
\end{itemize}

\subsection{Vasicek One-Factor Model (Basel II/III)}

Default is driven by a single systematic factor:
\[
Z_i = \sqrt{\rho}Y + \sqrt{1-\rho}\epsilon_i.
\]

Strengths:
\begin{itemize}
    \item closed-form solutions,
    \item regulatory standard.
\end{itemize}

Weaknesses:
\begin{itemize}
    \item oversimplified correlation,
    \item poor performance in crises.
\end{itemize}

\section{Comparative Summary of Models}

\begin{table}[h!]
\centering
\begin{tabular}{|p{1.5cm}|p{1.9cm}|p{1.5cm}|p{2.5cm}|p{3cm}|p{3cm}|}
\hline
\textbf{Model} & \textbf{Type} & \textbf{Mode} & \textbf{Inputs} & \textbf{Strengths} & \textbf{Weaknesses} \\
\hline
Credit Metrics & Bottom-up & MTM & Transition matrices, correlations & Captures migrations; detailed VaR & Correlation estimation; rating lag \\
\hline
KMV/ Merton & Structural & MTM & Equity, volatility, liabilities & Economic foundation; market PDs & Hard for private firms; default barrier issues \\
\hline
Credit Risk+ & Bottom-up & Default & PDs, exposures, sector factors & Efficient; minimal data & No migrations; Poisson assumption \\
\hline
Gaussian Copula & Bottom-up & Default & PDs, correlation matrix & Flexible dependence & Underestimates tail risk \\
\hline
Credit Portfolio View & Conditional & MTM & Macroeconomic variables & Great for stress tests & Calibration complexity \\
\hline
Vasicek (Basel) & Bottom-up & Default & PD, LGD, correlation & Regulatory closed forms & One-factor oversimplification \\
\hline
\end{tabular}
\caption{Comparative summary of major portfolio credit risk models.}
\end{table}

\section{Discussion}

Portfolio credit risk modelling has evolved from firm-value structural models to 
actuarial and copula-based approaches. Post-crisis, the emphasis shifted toward 
stress testing, macroeconomic linkage, and systemic stability. No single model 
captures all aspects of credit risk; thus, institutions often employ hybrid 
frameworks combining structural insights, dependence modelling, and 
macroeconomic sensitivity.

\section{Conclusion}

Portfolio credit risk models remain essential tools for risk management, pricing, 
and regulatory compliance. Understanding their assumptions, data requirements, 
and limitations is crucial for robust risk assessment. Future developments will 
likely integrate macro-financial interactions, machine learning, and stress 
scenario generation within portfolio credit risk frameworks.

\nocite{*}

\bibliographystyle{apalike}
\bibliography{references}

\end{document}
