\documentclass[12pt,a4paper]{article}

\usepackage{amsmath}
\usepackage{amssymb}
\usepackage{graphicx}
\usepackage{booktabs}
\usepackage{array}
\usepackage{geometry}
\usepackage{setspace}
\usepackage{hyperref}
\usepackage{natbib}

\geometry{margin=1in}
\setstretch{1.2}

\title{Portfolio Credit Risk Models: A Comprehensive Review}
\author{Your Name}
\date{\today}

\begin{document}

% ------------------------
% TITLE PAGE
% ------------------------
\begin{titlepage}
    \centering
    {\large University of Ljubljana\\[0.5cm]
    School of Economics and Business\par}
    \vfill
    {\LARGE \textbf{Portfolio Credit Risk Models:\\A Comprehensive Review}\par}
    \vspace{1.5cm}
    {\large Risk Management 2025/26\\
    Assignment No.\ 2\par}
    \vfill
    {\large
    Author: Alja Dostal, Manca Kavčič\\
    
    Date: \today\par}
\end{titlepage}

% ------------------------
% TABLE OF CONTENTS
% ------------------------
\tableofcontents
\newpage

% ------------------------
% ABSTRACT
% ------------------------
\begin{abstract}
This paper provides a comprehensive overview of portfolio credit risk modelling 
frameworks, including structural, reduced-form, actuarial, copula-based, and 
macro-econometric approaches. It presents a taxonomy of models, discusses their 
theoretical foundations, evaluates practical applications, and highlights key 
strengths and limitations. A comparative table summarises the major portfolio 
credit risk models widely used in practice.
\end{abstract}

% ------------------------
% MAIN TEXT
% ------------------------

\section{Introduction}

Credit risk—the potential loss arising from a counterparty’s failure to meet 
contractual obligations—is one of the central risks managed by financial 
institutions. While traditional credit analysis focuses on individual borrowers, 
portfolio credit risk extends this perspective by explicitly accounting for the 
joint behaviour of multiple obligors, default correlations, and systemic effects. 
Financial institutions rely on portfolio credit risk models to quantify expected 
loss (EL), unexpected loss (UL), and tail risk measures such as Value-at-Risk (VaR) 
and Expected Shortfall (ES), which are essential for economic capital allocation, 
regulatory compliance, pricing, and portfolio optimisation \citep{hull2018, mcniel2015}.

This paper provides a structured overview of the most important portfolio credit 
risk modelling frameworks and critically evaluates their assumptions, strengths, 
and limitations.

\section{Conceptual Foundations of Portfolio Credit Risk}

Portfolio credit losses can be expressed as
\[
L = \sum_{i=1}^{n} EAD_i \cdot LGD_i \cdot D_i,
\]
where $EAD_i$ denotes exposure at default, $LGD_i$ is loss given default, and $D_i$ 
is a default indicator variable. The primary challenge in portfolio credit risk 
modelling lies in capturing the joint distribution of defaults and exposures, as 
losses are rarely independent across obligors \citep{bluhm2016}.

Several fundamental modelling dimensions are commonly used to classify portfolio 
credit risk models. These include the distinction between top-down and bottom-up 
approaches, default-mode versus mark-to-market frameworks, conditional versus 
unconditional models, and structural versus reduced-form specifications. Each 
dimension reflects a different modelling philosophy and has important implications 
for practical applications.

\section{Typology of Portfolio Credit Risk Models}

\subsection{Top-Down and Bottom-Up Models}

Bottom-up models analyse credit risk at the level of individual obligors and then 
aggregate the results to the portfolio level. In this framework, each exposure is 
characterised by its probability of default, exposure at default, and loss given 
default, while portfolio risk emerges from the joint behaviour of defaults across 
obligors. Prominent examples include the Merton/KMV model, CreditMetrics, Gaussian 
copula models, and CreditRisk+ \citep{crouhy2000}. The main advantage of bottom-up 
models lies in their ability to provide detailed and granular risk information, 
which allows for portfolio optimisation, risk attribution, and sensitivity 
analysis at the level of individual borrowers. At the same time, these models 
require extensive obligor-level data and robust estimates of default correlations, 
which can be difficult to obtain in practice. As portfolio size increases, the 
computational burden of simulation-based bottom-up models also becomes significant.

In contrast, top-down models treat the credit portfolio as a single aggregate 
entity and directly model portfolio losses without explicitly modelling individual 
borrowers. Default probabilities are typically conditioned on macroeconomic or 
systematic risk factors, such as economic growth or unemployment rates. Such 
approaches are computationally efficient and are often used for stress testing and 
high-level scenario analysis, particularly when the focus lies on systemic risk 
rather than individual exposures. Their main limitation lies in their lack of 
interpretability and inability to provide borrower-level insights, which restricts 
their usefulness for detailed credit portfolio management \citep{wilson1997}.

\subsection{Default-Mode and Mark-to-Market Models}

Default-mode models focus exclusively on losses that occur at default. In these 
frameworks, credit risk is measured through default probabilities and loss given 
default assumptions, while changes in credit quality prior to default are ignored. 
Expected loss and unexpected loss are therefore driven entirely by the frequency 
and severity of default events. Examples include CreditRisk+, copula-based default 
models, and factor models such as the Vasicek framework \citep{vasicek2002}. These 
models are conceptually simple and computationally efficient, which makes them 
particularly attractive for regulatory capital calculations. However, their 
inability to capture rating migrations or spread dynamics limits their applicability 
for pricing credit-sensitive instruments.

Mark-to-market models extend the analysis by accounting for changes in credit 
quality before default, typically through rating migrations or credit spread 
movements. In this setting, portfolio losses may arise even if no default occurs, 
reflecting deterioration in creditworthiness. CreditMetrics is the most 
well-known example of this approach \citep{jpm1997}. While mark-to-market models 
provide a more comprehensive and economically meaningful view of credit risk, they 
require richer datasets, including rating transition matrices and market-based 
spread information, and involve more complex calibration and simulation procedures.

\subsection{Structural Models}

Structural credit risk models originate from the seminal work of 
\citet{merton1974}. In these models, default occurs when the value of a firm’s 
assets falls below a predefined default barrier, usually related to the book or 
market value of its liabilities. Asset values are assumed to follow a stochastic 
process, and default is an endogenous outcome of the firm’s capital structure and 
asset volatility. Structural models offer strong economic intuition and provide 
market-implied default probabilities that can respond quickly to new information. 
However, they often suffer from unrealistic default timing, as default can occur 
only at specific horizons, and from limited applicability to privately held firms 
for which market data on equity values and volatility are unavailable.

\subsection{Reduced-Form Models}

Reduced-form, or intensity-based, models treat default as an exogenous and 
unpredictable jump process governed by a stochastic default intensity 
\citep{jarrow1995}. In this framework, default arrives randomly according to a 
Poisson process, and the probability of default over a given horizon depends on 
the level and dynamics of the default intensity. These models are flexible and 
mathematically tractable and are widely used in the pricing of credit derivatives, 
where the timing of default is crucial. However, reduced-form models lack a direct 
economic interpretation of default events, as default is not explicitly linked to 
the firm’s balance sheet or asset dynamics.

\section{Major Portfolio Credit Risk Models}

\subsection{CreditMetrics}

CreditMetrics, developed by J.P.\ Morgan in 1997, is a market-to-market credit risk 
model that captures both default and migration risk through rating transition 
matrices, spread volatility, and correlated asset value dynamics \citep{jpm1997}. 
Each instrument in the portfolio is assigned a credit rating, and future rating 
states are simulated using empirical transition probabilities. In the event of 
default, losses are computed using historical recovery rates differentiated by 
seniority. Correlations between obligors are introduced through correlated asset 
returns, typically assumed to follow a multivariate normal distribution. The final 
portfolio loss distribution is obtained via Monte Carlo simulation. The main 
strength of CreditMetrics lies in its ability to integrate credit risk into a 
value-at-risk framework similar to market risk models, while its main weakness 
stems from the difficulty of estimating reliable correlations and transition 
probabilities.

\subsection{KMV / Merton Model}

The KMV model builds on Merton’s structural framework and introduces the concept of 
distance-to-default, defined as
\[
DD = \frac{\ln(V/A) + (\mu - \tfrac{1}{2}\sigma^2)T}{\sigma \sqrt{T}}.
\]
This measure captures how far a firm’s asset value is from the default boundary, 
expressed in units of asset volatility. The distance-to-default is then mapped to 
empirical default probabilities using large historical databases of default events. 
The model provides forward-looking and market-based default estimates and has been 
shown to anticipate changes in credit ratings. However, it requires high-quality 
market data and relies on simplifying assumptions regarding capital structure, 
asset volatility, and the timing of default \citep{merton1974}.

\subsection{CreditRisk+}

CreditRisk+ is an actuarial portfolio credit risk model that treats defaults as 
Poisson-distributed events with sector-specific risk factors \citep{bluhm2016}. 
Instead of modelling asset values or credit migrations, the model focuses on the 
frequency and severity of default losses. Correlation across obligors is introduced 
by grouping exposures into homogeneous sectors that share common systematic risk 
factors. Losses are modelled using exposure bands, and the overall portfolio loss 
distribution is obtained by aggregating losses across bands and sectors. Its 
computational efficiency and modest data requirements make CreditRisk+ attractive 
for large loan portfolios, although it ignores rating migrations and assumes 
simplified dependence structures.

\subsection{Gaussian Copula Models}

Gaussian copula models link individual defaults through a common dependence 
structure, typically represented as
\[
D_i = \mathbf{1}(Z_i < \Phi^{-1}(PD_i)).
\]
In this framework, default occurs when a latent variable falls below a threshold 
determined by the obligor’s marginal default probability. Dependence across 
obligors is captured through correlations among the latent variables. These models 
gained widespread popularity in structured credit markets due to their analytical 
convenience and flexibility. However, their inability to capture extreme tail 
dependence became evident during the global financial crisis, leading to systematic 
underestimation of joint default risk \citep{crouhy2000}.

\subsection{Credit Portfolio View}

Credit Portfolio View explicitly incorporates macroeconomic variables such as GDP 
growth, unemployment, and interest rates into the modelling of default 
probabilities. Default probabilities are estimated using econometric models, 
typically logit specifications, and vary systematically with the state of the 
economy. Portfolio losses are then simulated conditional on macroeconomic 
scenarios. This approach makes Credit Portfolio View particularly well suited for 
stress testing and scenario analysis. However, its top-down nature limits its 
ability to capture firm-specific credit risk, and its calibration requires long 
and reliable macroeconomic time series \citep{wilson1997}.

\subsection{Vasicek One-Factor Model}

The Vasicek one-factor model, which forms the basis of the Basel II and Basel III 
IRB capital framework, represents default risk as driven by a single systematic 
factor \citep{gordy2003}:
\[
Z_i = \sqrt{\rho} Y + \sqrt{1-\rho} \epsilon_i.
\]
In this framework, default correlations arise naturally from exposure to the common 
systematic factor. The model offers closed-form solutions for portfolio loss 
quantiles and provides a transparent regulatory standard. However, the assumption 
of a single risk factor oversimplifies correlation structures, particularly during 
periods of financial stress.

\section{Comparative Summary of Models}

\begin{table}[h!]
\centering
\begin{tabular}{|p{1.5cm}|p{2cm}|p{1.3cm}|p{2cm}|p{3cm}|p{3cm}|}
\hline
\textbf{Model} & \textbf{Type} & \textbf{Mode} & \textbf{Inputs} & \textbf{Strengths} & \textbf{Weaknesses} \\
\hline
Credit Metrics & Bottom-up & MTM & Transition matrices, correlations & Captures migrations; detailed VaR & Correlation estimation; rating lag \\
\hline
KMV/ Merton & Structural & MTM & Equity, volatility, liabilities & Economic foundation; market PDs & Requires market data; barrier issues \\
\hline
Credit Risk+ & Bottom-up & Default & PDs, exposures, sector factors & Efficient; minimal data & No migration; Poisson assumption \\
\hline
Gaussian Copula & Bottom-up & Default & PDs, correlations & Flexible dependence modelling & Underestimates tail risk \\
\hline
Credit Portfolio View & Conditional & MTM & Macro economic variables & Ideal for stress tests & Calibration complexity \\
\hline
Vasicek (Basel) & Bottom-up & Default & PD, LGD, $\rho$ & Closed forms; regulatory standard & One-factor oversimplifies correlation \\
\hline
\end{tabular}
\caption{Comparative summary of major portfolio credit risk models.}
\end{table}

\section{Discussion}

Portfolio credit risk modelling has evolved from economically intuitive structural 
models to actuarial and copula-based frameworks, as well as regulatory factor 
models. Following the global financial crisis, greater emphasis has been placed 
on stress testing and macroeconomic sensitivity. In practice, institutions often 
combine multiple models to compensate for individual limitations.

\section{Conclusion}

Portfolio credit risk models are indispensable tools in modern risk management. 
Each modelling framework embodies specific assumptions and trade-offs, making it 
essential to understand their limitations and appropriate use cases. Future 
developments are likely to further integrate macro-financial feedback mechanisms 
and advanced analytical techniques.

% ------------------------
% BIBLIOGRAPHY
% ------------------------

\nocite{*}

\bibliographystyle{apalike}
\bibliography{references}

\end{document}
